\chapter{Preparation}
\label{chapter:preparation}

Utilizing undergraduates in teaching positions is not without its risks.
\begin{quote}
    Undergraduates who are unclear on the material may cause confusion among their peers. In addition, not all undergraduates have the knowledge or maturity to successfully teach, assess, or mentor their peers, or understand conflict-of-interest situations. If poorly implemented or not properly supervised, this approach can place additional strain on course instructors. \cite{CSUndergraduateEnrollments}
\end{quote}
Preparation is especially important as the program expands in size and hires more undergraduate TAs to support large enrollment courses in both lower-division and upper-division courses.

\section{Introduction to Teaching Computer Science}
\label{section:cs370}

\begin{quotation}
    [CS 370: Introduction to Teaching Computer Science] is a course designed help aspiring teachers hone their teaching skills, become a part of the teaching community here at UC Berkeley, and expose them to the foundations of computer science pedagogy. Students in this class will receive first-hand experience through one-on-one tutoring and an enriched teaching knowledge through research-based pedagogical studies.

    CS 370 has three key components that distinguish it from other pedagogical courses. First, we cover student interactivity and teaching in one-on-one settings. This is applicable to all levels of teachers [\dots\unkern] since one-on-one interactions are a critical component of all teaching experiences. Next, we cover group teaching through in-class demonstrations, as mastering pacing and understanding the individualities of students in a group setting is key to being a successful TA. Last, we socratically discuss current issues in CS pedagogy, including atmosphere-related questions such as: underrepresentation, stigmas associated with computer science, the issue of prior experience, and how these factors heavily influence student learning.\footnote{\url{http://inst.eecs.berkeley.edu/~cs370/policies.html}}
\end{quotation}

Students are introduced to pedagogical concepts during an 80-minute seminar each week that includes discussion of ideas and reflection on their teaching experiences in small groups. Outside of the classroom, students read scholarly articles on the practice and theory of teaching computer science, host three, hour-long one-on-one tutoring sessions per week, and reflect on their tutoring as part of a weekly written assignment. Combining theory and practice together helps students learn and retain material, treating topics taught in class as a frame for questions brought up in self-reflections on their teaching experiences. Group discussions are facilitated by experienced TAs whose experience students identify with and more closely relate. To facilitate discussion, these weekly seminars are held in an active learning classroom with students seated around tables and facing each other rather than the front of a lecture hall. Between the weekly seminar, tutoring, tutoring preparation, tutoring reflection, and weekly assignments, CS 370 is a total commitment of 9 hours per week.

CS 370 was designed originally as a course to prepare and engage new teachers which influenced its decision to use one-on-one tutoring as the context for teaching practicum. Unlike other programs at peer institutions, a large number of aspiring undergraduate student-teachers take the course before they become TAs. This results in a diversity of students composed of first-year students who just recently took the courses they want to someday teach as well as older, second or third-year students, which makes for engaging conversations as their different experience levels provide greater opportunities to learn from each other. More-experienced student-teachers in the group and the experienced TA facilitators can chime in and provide nuanced viewpoints to questions less-experienced teachers might have about teaching one-on-one or leading small groups.

This design of CS 370 has a number of consequences which has made it particularly well-suited for preparing undergraduate TAs. First, the outline of topics includes concerns which are especially important for teaching at the undergraduate level such as diversity, unconscious bias, and tackling misconceptions. CS 370 is complemented by CS 375 which is geared toward a graduate student audience and, notably, includes coverage of topics such as developing course syllabi, exam problems or rubrics, and student surveys, all of which are tasks that concern head TAs responsible for the administrative component of a course but not necessarily teaching-focused TAs. CS 370, CS 375, as well as other pedagogy courses at peer institutions have also found success running the course in a workshop style with a significant portion of the materials presented at the beginning of the semester to maximize their effect on teaching, and then later fading away to more infrequent check-ins later in the semester \cite{Roberts:1995}.

\section{Mentoring at Scale}

Near-peer student mentors, such as the students who lead small-group sessions for CSM, are organized into a family system to prepare for their weekly group sessions and build community. Like TA families, mentors are grouped into families of about 6 mentors, each consisting of two experienced senior mentors and about four less-experienced junior mentors. In addition to providing feedback, checking-in, and bonding over social events, mentors also meet together regularly for one hour each week to prepare for the upcoming week's group sessions with mentees. Family meetings are a mix between active problem-solving, 3-minute teaching demonstrations, real-time critiques, and moments of written self-reflection. In these weekly family meetings, senior mentors lead and facilitate group discussions with junior mentors about the challenges and pitfalls of upcoming concepts, and assist mentors in personalizing their session to meet their mentees' needs. In order to make this hour effective, junior mentors prepare for the family meetings in advance by spending half an hour reviewing concepts in advance and preparing a mental outline of the lesson they have in mind for their session.

Most mentors only lead one or two group mentoring sessions. Since these mentoring sections only consist of 4--6 students each, for larger classes, over 100 sections are offered each week. In order to support this structure, CSM delegates the task of organizing mentors to the course coordinators, highly-experienced mentors who manage the entire operation. Course coordinators play a similar role as Section Leader Coordinators and Meta-TAs implemented at peer institutions \cite{Reges:2003, Roberts:1995, Reges:1988}. They hold weekly meetings with all of the senior mentors to prepare content for the family meetings and group mentoring sessions and assist the senior mentors in preparing for facilitating their own family meetings.

\section{Course-Specific Preparation}

Large classes make it harder for instructors to provide individual feedback to students. Likewise, large course staffs make it harder for instructors to provide personalized mentorship to their TAs. As course staffs have grown beyond 20, 30, 40, and even 50 TAs, several CS courses have begun grouping their course staff members into smaller families as well. Each TA family consists of 4--6 TAs with a mix of experienced and inexperienced teachers. As part of their preparation duties, TAs are occasionally expected to shadow and provide feedback to other family members to improve their teaching. Mirroring mentoring families, lead TAs check-in with their family members throughout the semester and organize occasional social outings with the entire group, building a community between undergraduate TAs.

In addition to the formal CS 370 pedagogy course and the more informal family system, undergraduate student-teachers also receive support and mentorship at the course level. The instructor of record and more-experienced TAs will often share their preparation materials, refine assignment guides, discussion walkthroughs, and other documents designed to support newer teachers. Discussions handouts are often reused between semesters so course staff share potential ways of teaching the concepts. Assignment guides provide answers to frequently-asked questions, identify common student bugs and their fixes, and suggest relevant connections to previous concepts and prerequisites to bridge knowledge gaps.

It is also common for course staff to run their own preparation sessions at the beginning of each semester to on-board new course staff members, set expectations, and provide course-specific guidance. Topics include preparing for discussion, lab, and office hours; modeling behavior and setting student expectations for the course's pace, format, and recommended learning strategy; course-specific resources and policies that need to be shared with students early in the semester; and upcoming changes for the current offering of the course. The course staff set four ground rules for one-on-one interactions in office hours:
\begin{enumerate}
    \item If you don't know what to do, ask.
    \item Be sensitive because learning computer science can be hard.
    \item Let the student drive.
    \item Do not give away the answer, if you can help it.
\end{enumerate}
These conversations are continued throughout the semester during weekly staff meetings where the entire course staff meets to make decisions on open administrative questions, give and receive feedback on new ideas or proposals, and plot out the next few weeks' content in the course. At the final meeting, time is set aside for course staff to reflect on the entire semester as a whole and determine where improvements can be made to assignments, teaching, policies, and the overall design of the course.

In addition, experienced TAs propose an idealized assignment help workflow to normalize lab and office hours expectations across the course staff. The goal of this workflow is to reduce the risk of providing too much assistance, which can harm students as they grow dependent on the guidance and are unable to solve problems on their own \cite{Smith:2017}. For programming courses, this workflow starts with understanding the question since students often miss important details when focused on the problem. The TA is directed to sit down beside the student, ensuring that their eye-lines match, and introduce themselves and learn the student's name. These practices help to build trust and rapport between the student and the TA, particularly if the student and the TA meet again in lab or office hours. The next step is to ask the student to describe their problem in their own words. This gives the TA time to skim the student's code and verify that the student's explanation matches their code, and later work with the student to identify the source and cause of the problem. After the student gains an understanding of the problem, the TA works with the student to formulate a plan to resolve the problem, and then gives the student time to solve it on their own. After about 10 minutes, during which the TA helps another student, the TA returns to check back in on the student's progress. These last few practices give the student space to work on the problem on their own and encourages them to build independence. Rather than sitting with the student and solving their problems for them, the TA's goal is to have the student in a better position to solve the problem independently.
