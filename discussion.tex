\chapter{Discussion}
\label{chapter:discussion}

While these methods have enabled CS courses at UC Berkeley to scale to meet both CS major and non-major student demand, the system of incentives, particularly the LSCS 3.3 GPA cap, strains relationships between instructors and students. Implementing student-friendly course policies, designing collaborative assignments, and encouraging students to take advantage of mentorship opportunities can significantly improve the student learning experience but is ultimately contrary to the message sent by the GPA cap. There is little room for failure: students who struggle in one or two of the introductory courses face an uphill battle to make it to the GPA cap. This is in spite of the limited evidence that, when given a second chance, students are able to make a remarkable improvement. In Spring 2016, students who were given the option to receive a failing grade in introductory Data Structures and retake the course in a later semester made an average improvement of +2.54 grade bins over the grade they would have received in Spring 2016.

The EECS Department has made significant gains in improving the gender diversity of its undergraduate student population, receiving recognition by the National Center for Women \& Information Technology (NCWIT)\footnote{\url{https://www.ncwit.org/2019-ncwit-extension-services-transformation-next-award-recipients}} as well as local news media\footnote{\url{https://www.mercurynews.com/2018/04/16/forget-techs-bad-bros-stanford-berkeley-boost-female-computing-grads/}} for its achievements. However, there is still much work to be done to encourage participation from a broader population of students. For some students, the time, energy, and stress necessary to meet the GPA cap makes the major unattractive. Other students may not have the confidence to pursue the major despite interest and academic preparation. In order to grow capacity while maintaining an inclusive student culture, it is important to take into account the entire system of incentives and punishments. Policies such as the GPA cap have ripple effects as student perceptions of the program on campus are shaped by its reputation of being highly rigorous, demanding, and stressful. Students may not feel comfortable if they see the program---including its faculty, staff, and students---as competitive in spite of their best efforts to design collaborative and supportive learning experiences. When a potential student's sensibilities do not line up with these perceived values, students may feel excluded from CS even if they could otherwise be successful computer scientists.

Even with technology and a large number of highly motivated support staff, faculty teaching load remains a significant burden, particularly for new tenure-track assistant professors who also need to balance their research output and tenure priorities with teaching. In particular, the faculty find larger enrollments have resulted in greater administrative workload, one that has not yet been fully displaced by head TAs despite all of the preparation and software in place to support them. Furthermore, courses that rely on individually-personalized projects or deep-feedback assignments do not easily fit into this framework of automation solutions. As a consequence, many faculty now choose to co-teach courses, which reduces workload but restricts course offerings.

Additionally, the use of some automation has the potential to impact student behavior in unexpected ways. The office hours queue, for example, compartmentalizes assignment help and student questions into individual tickets that are then resolved one-by-one. This model works in introductory courses due to the large number of TA office hours, and because assignments are typically scaffolded to help students make progress and facilitate efficient resolution in office hours. But the kinds of debugging and self-regulation practices acquired through these introductory CS office hours don't necessarily prepare students for upper-division coursework where the questions are more open-ended and the debugging processes much less clear.

Designing a CS program that scales requires cooperation from all levels of campus, including the students, staff, faculty, the department, the college, and the administration. While this report focuses on recent developments, a culture of innovation by undergraduate TAs has long existed in the EECS Department since their introduction in the 1980s, and by graduate students even before then. Each class of students is supported by the preceding class of students who teach section each week and serve as role models. Staff have worked closely with students and faculty to develop novel solutions to challenges of teaching CS at scale while advising triple the number of students from just a decade ago. Faculty have made sacrifices to teach at this scale, often teaching courses double their expected teaching load due to exploding enrollments. The College of Engineering, Graduate Division, and campus administration have supported the program by committing additional faculty slots, expanding advising support, and expanding course enrollments by funding additional TAs.

However, the program still faces a number of budgetary shortfalls as campus Temporary Academic Support (instructional support) has not kept up with the unprecedented growth of the program. Contributions from private and industry donors have enabled the department to continue funding more TAs, opening more sections, and reaching more students in spite of structural deficits and budget cuts at the university and state level. This external investment has fueled the development of automation solutions, support initiatives, and preparation processes that have made the UC Berkeley EECS Department a national model for teaching CS at scale.