\chapter{Support}
\label{chapter:support}

First popularized in the 1980s \cite{Reges:1988}, undergraduate students have been increasingly utilized in teaching positions at institutions of all sizes \cite{Roberts:1995, Reges:2003, Decker:2006, Dickson:2017, RetentionCS, Pon-Barry:2019}.
\begin{quote}
    Undergraduates have the potential to provide significant help and reduce the workload for graduate TAs and for faculty and other instructors. Undergraduates are typically paid per hour of effort, which can cost significantly less than graduate TAs. In addition, the shared experience of undergraduates may make them particularly attuned to understanding the problems and the challenges facing their peers around specific content. Furthermore, from a pedagogical perspective, peer teaching and evaluation can be valuable learning experiences in and of themselves, and can help empower students and build their confidence with the material. Finally, the undergraduate pool is larger than those of graduate students or faculty, and typically more diverse, presenting an opportunity for a more diverse set of instructors, which could contribute to a more inclusive culture. \cite{CSUndergraduateEnrollments}
\end{quote}

In the EECS Department at UC Berkeley, undergraduate teaching assistants (TAs) have been utilized since before 1987. One of the defining features of the undergraduate teaching program at UC Berkeley is the culture of student-directed innovation. Tools such as OK were developed by undergraduate teaching assistants to solve grading and feedback challenges. Situated within this existing infrastructure, the role of the instructor is part role model and part leader, with the goal of fostering a productive undergraduate teaching culture.

Students have multiple pathways to engage with the broader community beyond course staff. A constellation of extracurricular opportunities for students has developed over the past few years. Dedicated staff have created several services for the CS student community such as the CS Scholars program in which cohorts of 30 students take classes together for three semesters. With rising enrollments, more students than ever before are involved in EE and CS student organizations. The programs developed by staff and students for recruiting and retaining women in CS have been recognized by the National Center for Women \& Information Technology.\footnote{\url{https://www.ncwit.org/2019-ncwit-extension-services-transformation-next-award-recipients}}

\section{Undergraduate Teaching Assistants}

In the literature, it is common for the term Undergraduate Teaching Assistant to describe an undergraduate student teaching in a non-traditional role such as providing assistance to students in drop-in office hours or offering optional small-group or one-on-one tutoring. Notably, this definition refers to students who have a set of responsibilities that are distinct from graduate TAs. At UC Berkeley, undergraduate students hired as teaching assistants are responsible for the same set of duties as their graduate student counterparts: they teach weekly lab and discussion sections, grade assignments and exams, and assist with course delivery. These undergraduate students are officially hired under the title Graduate Student Instructor (GSI), though they are sometimes referred to as Undergraduate Student Instructors (UGSIs) to more accurately describe their academic standing.

EECS PhD students have a teaching requirement of 30 hours of service as a GSI, including 20 hours of service in an undergraduate course. However, the demand for TAs far exceeds the supply of graduate students. This is exacerbated by the fact that the vast majority of EECS PhD students are supported by research grants or fellowships that enable them to focus on their research. The hiring situation is particularly impacted in lower-division CS courses as graduate students often prefer to teach courses in their specific research area and taught by their faculty research advisors. As such, most introductory CS courses at UC Berkeley are taught with primarily undergraduate teaching assistants despite the fact that graduate TAs receive priority for these positions.

As the number of declared CS majors has increased, the demand for TAs in upper-division courses has also exploded beyond the supply of graduate TAs. Compared to peer institutions, the design of the undergraduate CS program makes it easier for the EECS Department to identify qualified undergraduates to staff upper-division courses and meet demand in spite of a shortage of interested graduate students. CS upper-division coursework is relatively flat with short prerequisite chains. Courses such as Introduction to Artificial Intelligence, Operating Systems, or Computer Security do not require any courses beyond the introductory course sequence. Students often satisfy the core introductory courses within two or three semesters so that, by the end of their second year, many students will have taken a couple upper-division CS courses that they can then teach over the remaining two years of their undergraduate degree program. Additionally, students are not required to complete a capstone project, leaving them more time to commit to coursework or extracurricular activities such as teaching or research.

In 2011, the largest introductory CS courses at UC Berkeley would hire 10 teaching assistants to serve classes with enrollments of about 350 students. Typical TA duties included both teaching and administrative responsibilities, usually split evenly across the entire course staff.
\begin{description}
\item[Teaching] Leading lab and discussion sections each week; holding weekly office hours; advising students; preparing for these teaching activities; and participating in weekly staff meeting.
\item[Administrative] Developing handouts, lab exercises, homeworks, and projects; grading assignments; handling accommodations for exceptional circumstances; managing announcements and student questions on the course forum; and proctoring and grading exams.
\end{description}

In recent years, however, grading and feedback tools such as Gradescope and OK have automated or streamlined many grading tasks. Tools have been developed to simplify traditionally expensive processes such as exam administration and assignment extensions. Furthermore, as course enrollments increase, the workload for certain aspects of course administration remain fixed. For example, the number of assignments and exams in the course is generally independent of the number of students enrolled in the course. In contrast, the teaching load grows linearly with respect to the number of students in the course.

Course staff composition has changed to reflect this new context. In a class of over 600 students with more than 20 TAs, there might be only a handful of 5--10 head TAs who are responsible for all of the course's administrative tasks. Instead of splitting tasks evenly, these 5--10 head TAs are each assigned one or two administrative responsibilities in addition to their regular teaching duties. There may be one or two TAs responsible for developing handouts, lab exercises, homeworks, and projects, which allows them to become domain experts in developing assignments for the course and maintaining a high quality of assignments with fewer bugs and greater consistency. This shift allows the remaining TAs to focus on teaching their students as effectively as possible. The number of these teaching-focused TA positions can be scaled at the same rate as course enrollment without significantly affecting course administration activities. Managing this greater number of teaching-focused TAs has become an administrative responsibility in and of itself so there may also be a head TA whose duty is to manage the course staff, communicate expectations, announce upcoming activities, and improve the quality of teaching.

\section{Center for Student Affairs}

The EECS Department served over 27,000 student enrollments across all course offerings during the 2018 academic year. Managing this number of students presents an administrative challenge for operating the program at scale, and can easily create feelings of anonymity among students, harming recruitment and retention efforts. The Center for Student Affairs (CSA), an EECS staff unit that provides several functions for undergraduate and graduate CS education, has developed a number of programs and solutions to tackle these challenges.

Starting Fall 2013, the CS Scholars Program,\footnote{\url{https://eecs.berkeley.edu/cs-scholars}} based on student retention theories of first year college students \cite{Tinto:1987, Terenzini:1980} and minority engineering students \cite{Treisman:1992}, is one such solution.
\begin{quote}
    CS Scholars is a first-year student support program intended to serve those from under-represented communities who have had little or no exposure to Computer Science. A learning community, CS Scholars integrates several components of support to meet the academic, social, and developmental needs of students intending to study Computer Science. Those components include:
    \begin{itemize}
    \item Cohort-style course discussions
    \item CS Scholars only seminars for personal and professional development
    \item Solidarity and community building activities
    \item Dedicated CS Scholars Advising
    \end{itemize}
\end{quote}
Data analysis has shown that the CS Scholars cohorts outperform students in the general population by 10--20\%, and students maintain a higher GPA than the overall class. In earlier cohorts, among students who identify as having no prior programming experience, CS Scholars had a 0.3 GPA advantage over non-scholars, and a greater difference for students who self-identified as female. The CSA-led EECS Resiliency Project is another retention initiative, which draws attention to stories from students and faculty who struggled with computer science at some point in their lives but persevered through those experiences of failure.

To diversify participation in and access to research experiences and graduate work in computer science, the CSA developed the Summer Undergraduate Program in Engineering Research at Berkeley (SUPERB),\footnote{\url{https://eecs.berkeley.edu/resources/undergrads/research/superb}} an NSF-funded Research Experience for Undergraduates (REU) program. Participants include junior and senior undergraduate students at Berkeley or elsewhere, and each participant receives faculty mentorship, graduate student support, and graduate school advising. 95\% of the students who participated in SUPERB continued to graduate school in STEM fields \cite{Alivisatos:2017}.

Due to the large number of students in the EECS major or considering declaring the LSCS major, most undergraduate advising is provided by professional EECS and LSCS major advisors. The advising staff assists with student questions and concerns including those related to the CS degree programs, coursework, undergraduate research, as well as students' broader plans and how they might fit into their life or career goals. The advising staff also manages a team of undergraduate peer advisors.

Getting into CS courses has become an often-cited grievance for undergraduates enrolled in universities, both large and small, across the nation. One of the CSA's functions is to coordinate between faculty and students to ensure that teaching supply is properly calibrated to meet enrollment demands, so the CSA has dedicated staff members for managing course scheduling and enrollment. Course capacity in the EECS Department is primarily limited by availability of classrooms and teaching assistants. Allocation of most discussion classrooms and large lecture halls involves close cooperation with central campus administrative staff. However, campus spaces do not provide enough capacity for all CS courses, so space often needs to be found within EECS-managed buildings. This is complicated by the fact that many spaces are already earmarked for strictly research purposes or strictly academic purposes. The challenge of efficiently allocating the remaining shared spaces is further exacerbated by enrollment growth as research functions compete with rapidly-growing academic functions. Increasing enrollments has also increased the number of teaching assistant hires, which has resulted in a significantly enlarged payroll that, unfortunately, does not increase at a rate sufficient to meet teaching needs, let alone match enrollment trends. Additionally, hiring more TAs requires additional coordination with campus training for first-time GSIs, as well as department-level and course-level preparation (\autoref{chapter:preparation}).

\section{Near-Peer Student Mentors}

As of 2018, there are 42 student organizations officially registered with the EECS Department, many of which host events and provide services to the broader CS community, such as:
\begin{description}
\item[Mentorship] Student organizations provide mentorship opportunities by hosting one-on-one or small-group mentoring sessions, blending academic support with a sense of community.
\item[Invited Speakers] External speakers and alumni give talks on topics including diversity in tech, overcoming adversity, and well-being, as well as workshops on bias, equity, and inclusion.
\item[Industry Events] With a student organization as their sponsor, employers can host info sessions, tech talks, or other events such as puzzle hunts or trivia nights to network with students.
\end{description}
One of the unusual features of the EECS Department is the amount and diversity of student-driven, near-peer mentorship opportunities available to students. In the near-peer mentor model, mentors are only a couple years more senior than their mentees. Near-peer mentoring ``provides younger students with a positive, inspiring experience to learning about computing from college near-peer mentors,'' and ``helps students feel like they belong in CS, especially if their mentors have backgrounds or experiences similar to their own'' \cite{RetentionCS}.

One such mentorship program is CS Kickstart.\footnote{\url{https://cs-kickstart.berkeley.edu}}
\begin{quote}
   CS Kickstart is a week-long program open to any incoming UC Berkeley students that introduces them computer science while meeting other computer science students and professionals. This program primarily targets women who are interested in the fields of science, technology, engineering, and math. Participants get hands-on experience in programming introducing them to the creativity and diversity of computer science. Participants also get the opportunity to visit tech companies in the Bay Area to see what life is like for computer scientists in industry. For several years it served 25 incoming students, but recently this doubled. It draws almost all of its support from industry and individual donors. \cite{Alivisatos:2017}
\end{quote}
The 2019 cohort will consist of about 50 participants. The program is organized by a group of undergraduate and graduate women in computer science, and is offered free to participants despite housing, transportation, and activity costs thanks to industry sponsors. As a result of the program, 96 percent of participants felt more prepared to take their first CS course at Berkeley, and 95 percent had a greater motivation to pursue computer science.

Once students are on campus, there are several student organizations forming communities around various identities or affinity groups, many of which offer mentorship programs of their own. Serving the woman-identifying EECS community is the Association of Women in Electrical Engineering and Computer Science (AWE).
\begin{quote}
    The AWE Mentorship Program provides a framework for EE and CS women to develop and sustain mentoring relationships by matching incoming students with upper division women. As new students, mentees connect with their mentors at the beginning of the school year, receiving personalized academic and social help when needed. Throughout the academic year, mentees receive advice, encouragement, information, and insight from experienced peers. Mentors, in turn, gain satisfaction and knowledge from guiding fellow students while fostering a sense of community.\footnote{\url{https://eecs.berkeley.edu/resources/undergrads/eecs/women/mentoring}}
\end{quote}
Similarly, the Society of Women Engineers provides mentorship to the broader community of women in all kinds of Engineering, and the more recent FEMTech student organization engages with the broader campus community by providing outreach and mentorship activities such as FEMTech Launch, which ``provides office hours, extra help, and weekly tutoring sessions specifically geared towards women and underrepresented minorities in lower level CS courses.''\footnote{\url{https://femtechberkeley.com/index.php/education/}} Honor societies such as Eta Kappa Nu (HKN) and Upsilon Pi Epsilon (UPE) offer free drop-in tutoring to the EECS undergraduate community across a majority of the undergraduate coursework.

Computer Science Mentors (CSM)\footnote{\url{https://csmentors.berkeley.edu}} is a student organization that, like other programs, offers academic support together with the community-building benefits of near-peer mentorship, but is offered at large scale, serving nearly 2,000 students per semester across 6 introductory computer science and electrical engineering courses. A typical mentoring group consists of 4--6 students and 1 near-peer mentor. The mentor facilitates student discussions and group work with a focus on mastery learning. Mentors adapt each session to meet the group's needs, drawing on additional examples to clarify concepts and build student confidence. Over the course of the semester, the mentor gets to know each student on an individual basis, and students grow more comfortable with each other too. The development of these relationships makes it easier for the mentor to keep in touch with their students by setting up individual check-ins in addition to the group sessions, sharing their experiences and study advice, and referring students to free tutoring services offered by other members of the EECS community. Participation in CSM small-group mentoring has been shown to have a significant positive association with exam scores. Organizing, preparing, and mentoring the mentors has become a challenge of its own (\autoref{chapter:preparation}).