\chapter{Introduction}
\label{chapter:introduction}

\begin{quote}
    Computer science classrooms are overflowing at colleges and universities across the United States. Enrollments are rising quickly, not only for majors, but also for non-majors who recognize the importance of computing skills in today's economy. This enrollment growth puts enormous pressure on computer science departments, which have not been able to expand to keep pace. \cite{Roberts:2018}
\end{quote}
In the decade between 2008 and 2018, CS departments across the United States have experienced double-digit undergraduate enrollment increases while ``the overall growth in teaching capacity woefully lags the growth in students [with] the vast majority of departments [reporting] increased difficulty in managing the situation'' \cite{TaulbeeSurvey2018}. From 2006 to 2015, the average number of CS majors in large departments (25 or more tenure-track faculty) increased from 341 to 970 and for small departments from 158 to 499 majors \cite{GenerationCS}. While growth varies between programs, the data make it clear that ``significant growth is under way at many institutions,'' and that ``the conditions exist for continued growth in the demand for CS and related jobs, degrees, and courses'' \cite{CSUndergraduateEnrollments}.

In this report, we present three strategies which have enabled the effective teaching of large enrollment CS courses at the University of California, Berkeley (UC Berkeley):
\begin{description}
    \item[Automation] The autograder infrastructure and online platforms which are now able to provide instant feedback, minimize manual grading, and deliver courses at scale.
    \item[Support] The expansion of student support networks through changes in teaching assistant responsibilities and the development of several near-peer mentoring communities.
    \item[Preparation] The expansion of undergraduate teacher preparation programs to meet the increased demand for student teachers.
\end{description}
These strategies support recommendations previously published by the Association for Computing Machinery \cite{RetentionCS}, Computing Research Association (CRA) \cite{GenerationCS}, the National Academies \cite{CSUndergraduateEnrollments}, and other research universities \cite{Maher:2015, Malan:2010, Porter:2013, Guo:2013, Hug:2015, Hug:2017, Reges:1988, Roberts:1995, Alvarado:2017, Minnes:2018, Kay:1998}. In light of the national CS capacity crisis and the increasing size of CS courses, this report identifies automation as the force that has driven subsequent changes in support and teacher preparation practices.

\section{National CS Capacity Crisis}

\begin{quote}
    Current pressures on computer science units are extremely difficult to manage and will also intensify if enrollments continue to grow. Institutional administrators need to work with computer science units to find sustainable approaches to meet the student demand, accounting for important factors such as (1) lack of space for classes and units, (2) academic support required, (3) the limited pool of qualified teaching faculty, (4) the goals and needs of nonmajors taking CS classes, (5) the effect of class size on the course experience, and (6) the desired retention of both students and faculty. \cite{GenerationCS}
\end{quote}
According to the 2017 CRA Enrollment Survey, ``66\% of the 134 responding doctoral-granting units reported that the enrollment growth is having a big impact (i.e., causing significant challenges) on their unit,'' with more than 50\% of doctoral-granting institutions citing 6 significantly increasing problems due to growing enrollments: classroom space shortages, insufficient numbers of faculty/instructors, insufficient numbers of teaching assistants (TAs), increased faculty workloads, office space shortages, and lab space shortages \cite{GenerationCS}.

In response to these challenges, more than 50\% of doctoral-granting institutions have already taken 4 actions to manage student enrollments: significantly increase class sizes, increase the number of academic year sections, increase summer offerings, and reduce low-enrollment classes. More than 65\% of doctoral-granting institutions have already taken 4 actions to increase teaching capacity: use undergraduate TAs and tutors, use more adjuncts or visitors as instructors, use graduate students as instructors, and increase the number of teaching faculty \cite{GenerationCS}. A survey of 78 CS professors from 65 different institutions identified the following three most common approaches for addressing the capacity crisis: (1) altering course offerings by increasing class sizes, offering more sections, and reducing elective offerings; (2) hiring more faculty and TAs; and (3) restricting access to classes, directing non-majors to other classes, and ``weeding out'' students \cite{Patitsas:2016}.

Research suggests that certain interventions can significantly affect the recruitment and retention of women and underrepesented minorities (URM) in CS \cite{Cohoon:2002, Babes-Vroman:2017, Rheingans:2018, Newhall:2014, Narayanan:2018, Lewis:2017, GenerationCS, RetentionCS, CSUndergraduateEnrollments, DiversityGapsCS}.
\begin{quote}
    The underrepresentation of women and people from groups underrepresented in computing raises concerns for a variety of reasons, including (1) issues of equity and fairness, (2) the economic and competitive imperative of ensuring a large and diverse U.S. workforce, (3) the fact that better solutions are developed by teams with a diversity of people and perspectives, and (4) the increasing interdependency between American democracy and the ability to understand and navigate the presentation of information through technology. \cite{RetentionCS}
\end{quote}
Course-level and department-level policies can directly affect which students pursue the major or have access to advanced coursework. More than 40\% of doctoral-granting institutions limit enrollments in high-demand courses, advise less-successful students to leave the major, and require that students are in a major or minor in order to enroll in an advanced course \cite{GenerationCS}. Even nominally objective policies such as restricting access to the major based on GPA can disproportionately disadvantage URM students \cite{DiversityGapsCS, RetentionCS}. Furthermore, ``imposing such restrictions makes the relationship between faculty and students adversarial, causing students to become more competitive and, in many cases, angry,'' with students concluding that they aren't wanted and perpetuating the idea that ``computer science [is] competitive and unwelcoming'' \cite{Roberts:2016, Patitsas:2014, Patitsas:2016}.

\begin{quote}
    In the face of increasing enrollments institutions would do well to take lessons from the past. The share of CIS [Computer and Information Science] and CS bachelor's degrees going to women decreased precipitously beginning in the mid-1980s, and again during the dot-com bust. These drops coincided with past peaks in CS degree production, suggesting that high-enrollment conditions or the actions taken by institutions in response to these surges may have contributed to the decrease in representation of women in undergraduate CS during these times. \cite{CSUndergraduateEnrollments}
\end{quote}
Indeed, many of the actions taken by universities today mirror the actions undertaken in the earlier enrollment surge in the 1980s, which included (1) increasing teaching loads and class sizes, (2) hiring more part-time and adjunct faculty, (3) retraining faculty from other disciplines, and (4) limiting enrollments and access to the major \cite{Curtis:1982}. Many departments have since adopted some of the recommendations cited in the 1982 report including diversifying academic opportunities by creating teaching-track faculty positions and using technology to make education more efficient.

CS education has only recently advanced to the national agenda \cite{StateofCS2018}, slowing the adoption of these ideas and practices. ``There are few researchers with CS education PhDs, and right now few or no active formal CS education PhD programs,'' \cite{CSforAll2018} stymieing the development of pedagogical methods and computer science education as a discipline. ``Teaching large computer science courses has become a more specialized endeavor,'' which grows capacity in impacted lower-division courses but results in an increase in student demand for upper-division courses without necessarily solving the underlying instructional bottleneck \cite{Roberts:2016}. There are simply not enough CS teachers. Furthermore, this capacity crisis is occurring at a time of institutional disinvestment due in part to ``administrators who are convinced that they [\dots\unkern] know when students will next lose interest'' but whose ``very decision ensures a capacity collapse'' \cite{Roberts:2016} in spite of evidence pointing to the opposite: ``While there will probably be fluctuations in the demand for CS courses, demand is likely to continue to grow or remain high over the long term'' \cite{CSUndergraduateEnrollments}.

\section{UC Berkeley Case Study}

This report presents a case study of three strategies for teaching CS at scale as developed in the Department of Electrical Engineering and Computer Sciences (EECS) at the University of California, Berkeley. Some historical context is necessary to understand the undergraduate CS education program that allowed the development of these strategies. The effectiveness of implementing them at other institutions will vary based on factors such as the institution's size and values \cite{CSUndergraduateEnrollments}.

There are two paths into the CS program at UC Berkeley:
\begin{description}
\item[EECS] The Electrical Engineering and Computer Sciences major in the College of Engineering, to which students apply directly in their application to the university, with a cohort of about 400 students matriculating in 2018.
\item[LSCS] The Computer Science major in the College of Letters and Sciences, where students are admitted into the college without declaring a major. Letters and Sciences students can declare the CS major after meeting the requirements for the major. During periods of high student demand and low supply, the LSCS declaration process can be very selective \cite{Roberts:2016, Alivisatos:2017}.
\end{description}
In 2019, the LSCS major is a capped major, admitting any student with an average 3.3 GPA across three introductory courses with an appeal process for students near the threshold. In 2018, about 800 students were accepted into the LSCS major. Based on current introductory CS course GPA trends, on expectation, about half of students who take the required courses will be eligible to declare the LSCS major, though these enrollments also include EECS majors and a large number of non-majors. Tracking students by their interest in the LSCS major, between 60--70\% of interested students successfully declare the LSCS major each year. However, this leaves an estimated 470 interested students unable to declare the LSCS major, including 170 women. This also leaves out students who do not even consider CS due to its reputation as a selective, capped major.

In 2018, between EECS and declared LSCS majors, the undergraduate CS program included over 3,200 majors, representing over 10\% of the university's undergraduate student population. In recent years, enrollment pressure has increased not only due to a growth in the number of majors, but also a growth in the number of courses students take per semester. The average number of upper-division EECS courses taken by a CS major throughout their undergraduate degree has recently increased from 5 courses to 7 courses. Students are taking more EECS courses to fulfill major requirements rather than electives offered by other departments. At the same time, the average time to graduation is only 7.89 semesters, as more students in the program are completing their degree in 3 or 3.5 years. Taken together, CS majors are choosing to take more upper-division CS courses in a shorter period of time, inflating enrollment pressure and demand for courses.

Several factors contribute to this growth. Most upper-division CS courses have a short prerequisite chain, usually only requiring the introductory CS sequence, so students can easily switch into another upper-division CS course if enrollment in their first-choice course is full. Furthermore, the program does not require a capstone project that, at many other institutions, introduces an individual advising requirement upon the faculty and consumes student attention in their final year of study. Department surveys show that students' post-graduation plans are increasingly focused on working in software engineering roles, so students value the technical expertise gained from taking technical, CS courses over breadth or personal interest courses.

Demand from non-majors has also increased. Despite the fact that enrollment preference is given to students in the CS program, an increasing number of non-majors are enrolling in upper-division CS courses. Two of the most popular courses (Introduction to Artificial Intelligence; Efficient Algorithms and Intractable Problems) enrolled about 25\% non-majors in 2018. Adjacent major programs including Data Science; Cognitive Science; Applied Mathematics; Engineering Mathematics and Statistics; Statistics; and Industrial Engineering and Operations Research either explicitly require or credit certain upper-division CS courses towards their undergraduate major degrees. This increase in non-major interest in CS courses mirrors the broader, national trend.

It is this context of external and internal demand for computer science that foreshadowed the Data Science undergraduate program, the ``fastest growing program in the history of Berkeley,'' \cite{Alivisatos:2017}.
\begin{quote}
    Berkeley's Data Science education program aims at a comprehensive curriculum built from the entry level upward to meet students' varied needs for data fluency. It includes a diverse constellation of connector courses that allow students to explore real-world issues related to their areas of interest and continues with intermediate and advanced courses that enable them to apply more complex concepts and approaches.\footnote{\url{https://data.berkeley.edu/education}}
\end{quote}
The Division of Data Sciences connects the School of Information, the EECS Department, the Statistics Department, the Berkeley Institute for Data Science (BIDS), and faculty, staff, and students from across campus. Introductory data science courses have been developed with lessons learned from introductory computer science \cite{Swamy:2018}, and upper-division courses are also designed to scale. Starting as a pilot course with 100 students in Fall 2015, enrollment in the introductory data science course, The Foundations of Data Science, reached 1,500 students in Spring 2019, exceeding enrollments in introductory CS that semester. Core data science courses are commonly co-taught by Statistics and Computer Science faculty while connector courses are offered by many departments across campus to meet the diversity in demand for computational literacy and data skills.

\subsection{Course Format}

The typical introductory computer science course uses the following model:
\begin{description}
\item[Lecture] 3 hours per week introducing concepts to the entire class, led by the instructor.
\item[Lab] 1--2 hours per week of hands-on exploration activities, led by a TA, with around 30 students.
\item[Discussion] 1--2 hours per week of group problem-solving, led by a TA, with around 30 students.
\item[Office Hours] A drop-in space for students to ask questions and get help with course concepts and assignments, normally offered on a regular basis by the instructors and TAs. 
\end{description}
The typical upper-level computer science course consists of 3 hours of lecture and 1 hour of discussion section per week. The EECS Department has experimented with other course formats as well. Data Structures and Programming Methodology is offered during summer session in lab-centric instruction format that consists of 1 hour of lecture and 6 hours of lab per week \cite{Titterton:2010}.

In part due to a shortage of large lecture halls, almost all CS courses have begun webcasting lecture. The campus information technology group records live lecture and posts the video online a few hours afterwards. Many students prefer webcasts over live lecture as they can speed-up, slow-down, pause, and rewind the video, so live lecture attendance in large courses rarely exceeds one third of the true class enrollment by the middle of the semester. Furthermore, lab and discussion section attendance is often not mandatory. Students are typically encouraged to participate, and the course policies may provide incentives for attendance, but attendance is rarely required.

As a consequence, there is a significant number of students enrolled in CS courses who rarely attend lecture but still learn all of the course content by watching webcasts. In 2018, this format was officially adopted by Introduction to Database Systems, which was offered to matriculating UC Berkeley students with the regular lecture sessions replaced entirely by professional recordings on the same content by the professor. To keep students on track, the course expects them to submit short, weekly quizzes on basic lecture concepts. Students are encouraged to attend discussion sections and office hours to clarify concepts from the webcast, build problem-solving skills, and collaborate with other students in the course. While relying on online resources frees demand for resources such as seating in lecture, these students often utilize other components of the course such as discussion, lab, office hours, study groups, and tutoring where learning occurs in smaller group environments. Scaling capacity in these activities has become an increasingly important focus for the department, a theme that is revisited throughout this report.