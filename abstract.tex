\begin{abstract}
    \noindent
    Over the past decade, undergraduate Computer Science (CS) programs across the nation have experienced an explosive growth in enrollment as computational skills have proven increasingly important across many domains and in the workforce at large. Motivated by this unprecedented student demand, the CS program at the University of California, Berkeley has tripled the size of its graduating class in five years. The first two introductory courses for majors, each taught by one faculty instructor and several hundred student teachers, combine to serve nearly 2,900 students per term. This report presents three strategies that have enabled the effective teaching, delivery, and management of large-scale CS courses: (1) the development of autograder infrastructure and online platforms to provide instant feedback with minimal instructor intervention and deliver the course at scale; (2) the expansion of academic and social student support networks resulting from changes in teaching assistant responsibilities and the development of several near-peer mentoring communities; and (3) the expansion of undergraduate teacher preparation programs to meet the increased demand for qualified student teachers. These interventions have helped both introductory and advanced courses address capacity challenges and expand enrollments while receiving among the highest student evaluations of teaching in department history. Implications for inclusivity and diversity are discussed.
\end{abstract}